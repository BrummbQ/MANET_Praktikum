\documentclass[12pt,a4paper,titlepage]{article}
\usepackage[utf8]{inputenc}
\usepackage[german]{babel}
\usepackage{amsmath}
\usepackage{amsfonts}
\usepackage{amssymb}
\usepackage{graphicx}
\usepackage{listings}
\usepackage[left=2cm,right=2cm,top=2cm,bottom=2cm]{geometry}
\title{Aufgabe 1}
\begin{document}
\title{Praktikum Mobile AdHocNetze\\
	Aufgabe 1: Experimente im Netzwerk-Simulator NS-3}
\author{
Bearbeiter: \and
	Andy Labitzke, tesgggggggggggggggggggggggggggggggt@mail.com \and 
	Gregor T, wire@gdggggggggggggggggggggggggggdg.cd}
\date{04.11.2012}
\maketitle
\setcounter{page}{1}
\appendix

\section*{Inhalt}
In diesem Dokument beschreiben wir die Lösung der 1. Praktikumsaufgabe. Die Gliederung orientiert sich an den einzelnen Unteraufgaben.

\setcounter{section}{0}

\section{Laden und kompilieren des NS-3 Package}

Text mehr zeilig envene Text mehr zeilig envene Text mehr zeilig envene Text mehr zeilig envene Text mehr zeilig envene Text mehr zeilig envene Text mehr zeilig envene Text mehr zeilig envene Text mehr zeilig envene Text mehr zeilig envene Text mehr zeilig envene Text mehr zeilig envene Text mehr zeilig envene Text mehr zeilig envene Text mehr zeilig envene Text mehr zeilig envene Text mehr zeilig envene 

\section{Funktionsweise kennenlernen}

Beispiel für Quellcode:

\lstinputlisting
	[caption={Ein kleines Programm in Java}\label{lst:javaclass}, %ref verwenden
		captionpos=b,
		language=bash] %oder TeX für plain
	{cwnd.plt}
 
 


\section{Eigenes Testszenario}

Beispiel für Gnuplot-Eps:

\begin{figure}[h]
	\centering
	\input{sixth}
	%\includegraphics[width=0.5\textwidth]{../output/cwnd.png}
	\caption{Titel der Grafik}
	\label{labelname}
\end{figure}



\section{Auswertung der Ergebnisse}

Text

\end{document}